% Created 2013-03-29 Fri 17:20
\documentclass[a4paper]{article}
\usepackage[utf8]{inputenc}
\usepackage[T1]{fontenc}
\usepackage{fixltx2e}
\usepackage{graphicx}
\usepackage{longtable}
\usepackage{float}
\usepackage{wrapfig}
\usepackage{soul}
\usepackage{textcomp}
\usepackage{marvosym}
\usepackage[nointegrals]{wasysym}
\usepackage{latexsym}
\usepackage{amssymb}
\usepackage{hyperref}
\tolerance=1000
\usepackage{fontspec}
\usepackage{xunicode}
\usepackage{xltxtra}
\usepackage{xeCJK}
\usepackage{listings}
\usepackage{xcolor}
\usepackage{fancyhdr}
\usepackage{fancybox}
\usepackage{comment}
\usepackage{enumerate}
\usepackage{colortbl}
\usepackage{framed}
\usepackage{amsmath}
\usepackage{algorithm}
\usepackage{algorithmic}

\setmainfont{Times New Roman}
\setmonofont{Courier New}
\setCJKmainfont[BoldFont=YouYuan]{SimSun}
\setCJKfamilyfont{song}{SimSun}
\setCJKfamilyfont{msyh}{微软雅黑}
\setCJKfamilyfont{fs}{FangSong}

\lstset{frame=single}

% new and renew command
\newcommand{\reffig}[1]{Figure~\ref{#1}}
\newcommand{\reftbl}[1]{Table~\ref{#1}}
\renewcommand{\contentsname}{目录}
\renewcommand{\baselinestretch}{1.2}

% In case you need to adjust margins:
\topmargin=-0.0in      %
\evensidemargin=0.5in     %
\oddsidemargin=0.5in      %
\textwidth=5.5in        %
\textheight=8.5in       %
\headsep=0.25in         %

\providecommand{\alert}[1]{\textbf{#1}}

\title{第五周工作报告}
\author{丘骏鹏}
\date{2013-03-29 Fri}
\hypersetup{
  pdfkeywords={},
  pdfsubject={},
  pdfcreator={Emacs Org-mode version 7.8.11}}

\begin{document}

\maketitle



\section{主要工作内容}
\label{sec-1}

这一周主要开始搭建程序的基本框架和实现数据的一些预处理。
\subsection{简单的数据统计}
\label{sec-1-1}

首先对数据做了一些基本的统计:

\begin{center}
\begin{tabular}{lrrr}
           &  blog.zip  &  news.zip  &  other.zip  \\
\hline
 文件个数  &     59998  &     81561  &     183635  \\
\end{tabular}
\end{center}



新浪博客的目录页和详细页可以用URL区分,比如某个博主的目录页为
\href{http://blog.sina.com.cn/u/1439351555}{http://blog.sina.com.cn/u/1439351555}
,他的某篇文章的URL格式为\\
\href{http://blog.sina.com.cn/s/blog_55cac30301016yb1.html}{http://blog.sina.com.cn/s/blog\_55cac30301016yb1.html}
。因此对于博客数据可以用URL
正则进行过滤。通过shell命令进行统计后,得到的结果为:
blog中不带html后缀的文件有23430,带有html后缀的文件有36568。可以初步判断blog数据
中有用的详细页数据为36568。

blog数据中还有少量的404页面,不符合上面的url组成规律的大部分是404页面。
我用简单的''404 Not Found''字符串进行过滤,以下的shell命令:

\lstset{extendedchars=false,basicstyle=\ttfamily\footnotesize,escapechar=`,breaklines,language=shell}
\begin{lstlisting}
ls blog/ | xargs -I{} grep "404 Not Found" -c {} | awk '{sum+=$1};END{print sum}'
\end{lstlisting}
结果为174,即有174个没用的404页面。这一步也可以先排除一些没用的404页面。(当然,
如果严格来说包含“404 Not Found”的页面不一定就是404页面,但是这个简单的筛选方法
对于这个实验来说应该够了)

结合以上方法可以做一次粗粒度的筛选,然后将可用的实验页面初步筛选出来。

我从blog和news中各抽取了1000个文件作为样本,用于测试使用。初步打算先用blog中的数
据进行后续的实验。
\subsection{实现}
\label{sec-1-2}

\begin{itemize}
\item 实现语言:
  打算采用Java+Scala,\href{http://en.wikipedia.org/wiki/Scala_%28programming_language%29}{Scala} 是JVM上的静态类型语言,可以和Java之间无缝操作,
  支持面向对象和函数式等编程范式。我之前写过Java+Scala的大作业,对两者都比较熟悉。
\item 关于实现方面的一些库的选择:
\begin{itemize}
\item 关于字符集探测库:上网搜了一些相关的库和相应的评价,决定选择\href{http://site.icu-project.org/}{icu4j}。这个库目前仍
    在活跃开发中,对各个字符集的支持很成熟。目前的测试结果来看可以对下载的HTML的字
    符集进行正常进行分辨。
\item HTML parser:目前初步决定采用\href{http://jsoup.org}{Jsoup},写了一些基本的程序进行初步测试,比较符合需
    求。
\item 日志系统:用的是twitter包装的util-logging包,是在java.util.logging上的一个简单
    包装。
\item 配置文件读取:基于java实现的一个配置文件读取库,支持java原生的properties文件格
    式、JSON格式和HOCON(Human-Optimized Config Object Notation)格式的配置文件。维护
    者是\href{http://typesafe.com}{http://typesafe.com},项目地址:\href{https://github.com/typesafehub/config}{https://github.com/typesafehub/config}
\end{itemize}
\item 目前已经实现的部分:
  目前写好了简单的预处理部分,包括先将一些不关心的标签去掉,如
  script,style,link,br,img,strong,em,font。这样可以先将文档的结果简化,后期DOM的
  解析会快,因为DOM解析一般会比较耗内存,而且速度较慢,因此这些预处理是必要的。
  另一方面,从我们需要关心的模板细度来说,我们也不需要这种细粒度的标签。不过这部
  分还需要不断修改验证,可以多去掉一些标签,但是不能将一些必要的去掉了。

  同时系统的日志系统和配置文件解析部分也已经基本完成。
\end{itemize}
\section{下周工作}
\label{sec-2}

下一周打算将网页的相似度计算和聚类部分完成,先采用一些基于path和tag的简单算法计
算相似度。

\end{document}