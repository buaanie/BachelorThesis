% Created 2013-03-05 Tue 19:11
\documentclass[a4paper]{article}
\usepackage[utf8]{inputenc}
\usepackage[T1]{fontenc}
\usepackage{fixltx2e}
\usepackage{graphicx}
\usepackage{longtable}
\usepackage{float}
\usepackage{wrapfig}
\usepackage{soul}
\usepackage{textcomp}
\usepackage{marvosym}
\usepackage[nointegrals]{wasysym}
\usepackage{latexsym}
\usepackage{amssymb}
\usepackage{hyperref}
\tolerance=1000
\usepackage{fontspec}
\usepackage{xunicode}
\usepackage{xltxtra}
\usepackage{xeCJK}
\usepackage{listings}
\usepackage{xcolor}
\usepackage{fancyhdr}
\usepackage{fancybox}
\usepackage{comment}
\usepackage{enumerate}
\usepackage{colortbl}
\usepackage{framed}
\usepackage{amsmath}
\usepackage{algorithm}
\usepackage{algorithmic}

\setmainfont{Times New Roman}
\setmonofont{Courier New}
\setCJKmainfont[BoldFont=YouYuan]{SimSun}
\setCJKfamilyfont{song}{SimSun}
\setCJKfamilyfont{msyh}{微软雅黑}
\setCJKfamilyfont{fs}{FangSong}

\lstset{frame=single}

% new and renew command
\newcommand{\reffig}[1]{Figure~\ref{#1}}
\newcommand{\reftbl}[1]{Table~\ref{#1}}
\renewcommand{\contentsname}{目录}
\renewcommand{\baselinestretch}{1.2}

% In case you need to adjust margins:
\topmargin=-0.0in      %
\evensidemargin=0.5in     %
\oddsidemargin=0.5in      %
\textwidth=5.5in        %
\textheight=8.5in       %
\headsep=0.25in         %

\providecommand{\alert}[1]{\textbf{#1}}

\title{第一周进展报告}
\author{计92 丘骏鹏 2009011282}
\date{2013-03-05 Tue}
\hypersetup{
  pdfkeywords={},
  pdfsubject={},
  pdfcreator={Emacs Org-mode version 7.8.03}}

\begin{document}

\maketitle



\section{每天进展}
\label{sec-1}

\begin{itemize}
\item 周二:安装实验室电脑,配置基本编程环境
\item 周三周四:阅读史兴的论文。有以下几个问题:
\begin{enumerate}
\item 关于data records的检测方法(基于桶)好像有鲁棒性不够的问题,若出现噪声标
       签会产生问题
\item ATree的构造方法还有一些细节方面感觉论文未详细论述
\end{enumerate}
除此之外,对于论文中提到的一些技术,如Hadoop,在网上自学了关于HFS、HBase等知
    识以及简单的MapReduce程序编写。其他的比较细的东西包括Google Protocol Buffers
    以及一些HTML清洗库(HTML Tidy,lxml.html模块等)。
\item 周五:从图书馆和网上找了有关dom tree clustering的相关资料。主要阅读了《XML挖
    掘: 聚类、分类与信息提取》一书,了解了一些计算两个XML文档相似性和进行聚类的
    方法。
\end{itemize}
\section{对工作的理解}
\label{sec-2}

主要工作分为两部分,一部分是从网页中分类出所需要的网页,一部分是对网页进行聚类,
然后生成模板进行提取。
\begin{itemize}
\item 第一部分工作我认为主要可以通过url来进行判断,一般目录页和内容页的url有较大差别,
  这种方法会非常有效率。如果这种方法不行,比如已有的网页的url全是随机生成的,是可
  以通过一些简单的特征,比如文本元素的内容长度进行筛选。这部分复杂度应该不能做得
  很高。
\item 第二部分工作是此次毕设的主要部分。主要的问题应该是“如何对Dom Tree进行聚类,并
  从每个类别中提取出模板”(有无理解错误?)。这部分还在阅读相关的书籍和论文,打
  算结合这周阅读论文的成果在这周的报告中给出。
\end{itemize}
\section{问题}
\label{sec-3}

\begin{itemize}
\item 目前有的资源有哪些?我看了上次您发给我的软院那个同学的东西,好像只是调研了Java
  的各个库如何抓取用了ajax的网页,应该只是和写网络爬虫那一方面相关的。目前我们有
  没有相关的网页,是需要自己抓取,还是已经存好了?如果已经有的话,如何获取使用?
  如果需要自己抓取数据或者预处理起来还有很多繁琐的细节,我觉得也可以和读论文同步
  展开。
\end{itemize}

\end{document}