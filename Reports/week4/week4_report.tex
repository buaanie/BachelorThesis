% Created 2013-03-22 Fri 18:31
\documentclass[a4paper]{article}
\usepackage[utf8]{inputenc}
\usepackage[T1]{fontenc}
\usepackage{fixltx2e}
\usepackage{graphicx}
\usepackage{longtable}
\usepackage{float}
\usepackage{wrapfig}
\usepackage{soul}
\usepackage{textcomp}
\usepackage{marvosym}
\usepackage[nointegrals]{wasysym}
\usepackage{latexsym}
\usepackage{amssymb}
\usepackage{hyperref}
\tolerance=1000
\usepackage{fontspec}
\usepackage{xunicode}
\usepackage{xltxtra}
\usepackage{xeCJK}
\usepackage{listings}
\usepackage{xcolor}
\usepackage{fancyhdr}
\usepackage{fancybox}
\usepackage{comment}
\usepackage{enumerate}
\usepackage{colortbl}
\usepackage{framed}
\usepackage{amsmath}
\usepackage{algorithm}
\usepackage{algorithmic}

\setmainfont{Times New Roman}
\setmonofont{Courier New}
\setCJKmainfont[BoldFont=YouYuan]{SimSun}
\setCJKfamilyfont{song}{SimSun}
\setCJKfamilyfont{msyh}{微软雅黑}
\setCJKfamilyfont{fs}{FangSong}

\lstset{frame=single}

% new and renew command
\newcommand{\reffig}[1]{Figure~\ref{#1}}
\newcommand{\reftbl}[1]{Table~\ref{#1}}
\renewcommand{\contentsname}{目录}
\renewcommand{\baselinestretch}{1.2}

% In case you need to adjust margins:
\topmargin=-0.0in      %
\evensidemargin=0.5in     %
\oddsidemargin=0.5in      %
\textwidth=5.5in        %
\textheight=8.5in       %
\headsep=0.25in         %

\providecommand{\alert}[1]{\textbf{#1}}

\title{第四周工作报告}
\author{计92 丘骏鹏 2009011282}
\date{2013-03-22 Fri}
\hypersetup{
  pdfkeywords={},
  pdfsubject={},
  pdfcreator={Emacs Org-mode version 7.8.03}}

\begin{document}

\maketitle



\section{工作内容}
\label{sec-1}

\begin{itemize}
\item A Short Survey of Document Structure Similarity Algorithms。这篇文章总结了已有
  的几种计算HTML文档结构相似性的方法,提出了Path Shingles的计算方法,文章中还提到
  了几种计算方法的若干实验结果,也可以当作本实验的参考。
\item Clustering Template Based Web Documents。这篇论文讨论一种优化的树的编辑距离计算
  方法和几个基于Path和Tag的相似度计算方法,同时还有将这些计算方法和几种聚类方法结
  合以后的实验效果,论文中的一些评价方法可以借鉴。
\item Detecting structural similarities between XML document。这篇文章只是简单的浏览,
  主要讲的是如何用傅里叶变换进行相似度计算。结合其他引用了这篇文章的论文来看,此
  方法比较复杂,效果却不一定很好,因此暂时不考虑使用。
\item 其他的大部分时间在准备、修改和完善开题报告
\end{itemize}
  
\section{下周计划}
\label{sec-2}

主要是开始写代码,将数据的输入方式和预处理做好,同时初步实现网页过滤的功能。可能
在实现过程中还要继续看一些论文,研究其中的细节。根据代码实现的进度再灵活调整工程
进度。

\end{document}