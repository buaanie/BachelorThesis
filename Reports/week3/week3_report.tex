% Created 2013-03-15 Fri 17:05
\documentclass[a4paper]{article}
\usepackage[utf8]{inputenc}
\usepackage[T1]{fontenc}
\usepackage{fixltx2e}
\usepackage{graphicx}
\usepackage{longtable}
\usepackage{float}
\usepackage{wrapfig}
\usepackage{soul}
\usepackage{textcomp}
\usepackage{marvosym}
\usepackage[nointegrals]{wasysym}
\usepackage{latexsym}
\usepackage{amssymb}
\usepackage{hyperref}
\tolerance=1000
\usepackage{fontspec}
\usepackage{xunicode}
\usepackage{xltxtra}
\usepackage{xeCJK}
\usepackage{listings}
\usepackage{xcolor}
\usepackage{fancyhdr}
\usepackage{fancybox}
\usepackage{comment}
\usepackage{enumerate}
\usepackage{colortbl}
\usepackage{framed}
\usepackage{amsmath}
\usepackage{algorithm}
\usepackage{algorithmic}

\setmainfont{Times New Roman}
\setmonofont{Courier New}
\setCJKmainfont[BoldFont=YouYuan]{SimSun}
\setCJKfamilyfont{song}{SimSun}
\setCJKfamilyfont{msyh}{微软雅黑}
\setCJKfamilyfont{fs}{FangSong}

\lstset{frame=single}

% new and renew command
\newcommand{\reffig}[1]{Figure~\ref{#1}}
\newcommand{\reftbl}[1]{Table~\ref{#1}}
\renewcommand{\contentsname}{目录}
\renewcommand{\baselinestretch}{1.2}

% In case you need to adjust margins:
\topmargin=-0.0in      %
\evensidemargin=0.5in     %
\oddsidemargin=0.5in      %
\textwidth=5.5in        %
\textheight=8.5in       %
\headsep=0.25in         %

\providecommand{\alert}[1]{\textbf{#1}}

\title{第三周报告}
\author{丘骏鹏}
\date{2013-03-15 Fri}
\hypersetup{
  pdfkeywords={},
  pdfsubject={},
  pdfcreator={Emacs Org-mode version 7.8.11}}

\begin{document}

\maketitle



\section{这周的工作}
\label{sec-1}

前半周的工作已经在上一个报告中给出了,这里将上一份报告的对应部分复制过来,另外汇报一
下下半周的工作。

\begin{itemize}
\item 阅读了TEXT: Automatic Template Extraction from Heterogeneous Web Pages。这篇论
  文将DOM Tree用路径进行表示,用MDL(Minimun Description Length Principle)作为标
  准进行聚类。由于直接计算MDL复杂度很高,该论文提出一种扩展的MinHash算法,用于近
  似估计MDL,提高算法效率,
\item 阅读了黄老师给的Webpage Understanding: Beyond Page-Level Search。里面主要介绍的
  是如何将webpage understanding拆分成几个子任务。其中提到了一种分割HTML网页的
  VIPS(VIsion-basd Page Segmentation)方法,即将原网页按照视觉区域的分块解析成一
  个vision tree来进行表示。
\item \href{http://code.google.com/p/cx-extractor/}{http://code.google.com/p/cx-extractor/} 上的工具和文档。该工具主要就利用了HTML
  文档行块的分布来提取正文,方法和实现都很简单。
\item VIPS: a Vision-based Page Segmentation Algorithm. 这篇论文讲解的是VIPS的实现。
  方法比较复杂,由于文章比较老(2003),网页的设计有很大的变化,论文中列出的一些启
  发规则可能要有一定的修改。此外,我找了一下,作者没有公开实现的代码。
\item 阅读了中文文献《基于视觉的 Web 页面分块算法的改进与实现》,是一个对VIPS算法的改
  进。主要关注的HTML中的<table>标签,将DOM Tree变成只有<table>标签的树(实际上对
  于<div>是一样的,现在网页大部分已经采用DIV+CSS的布局方法),同时不需要用规则判
  断节点是否可分,直接提取每一层的<table>标签子节点,后面还有对各个语义块进行合
  并的过程。
  
  我原来的想法是有没有可能在vision-tree的层次上做聚类,相同模板的网页视觉布局大概
  也会相同。但是现在网页设计有一些变化,布局信息更多放在CSS中,因此如果网页设计者
  将HTML的布局和内容分开得比较好的话,HTML源代码中大部分会是内容,布局信息会很少,
  而这个方法主要是依靠tag里的一些视觉相关的属性值进行判断,加上这个系统比较复杂,
  我没找到现成的实现(好像郝老师说有类似实现),可能可行性不高。
\end{itemize}
\section{我的想法}
\label{sec-2}

  从目前的调研结果来看,我觉得TEXT: Automatic Template Extraction from
  Heterogeneous Web Pages这篇论文和我要做的工作比较接近,主要关注的也是从异构的网
  页中抽取模板。他的基本特征是基于HTML的DOM Tree的路径集合,改进的地方在于提出了
  一个近似算法(Extended MinHash)对计算进行加速。关于Extended MinHash算法的很多
  细节我当时没有全部看懂,我打算再认真看看论文的那一部分。

  现阶段我的想法是基于vision-tree的方法可能不太可行,还是基于原始的DOM Tree进行聚
  类。模板的表示基于路径会好一些,但聚类时候用的特征不一定局限于路径,可以结合标
  签加上文本内容特征。另外,我之前看的《XML挖掘:聚类、分类和信息提取》书中的利用
  WordNet进行路径的词义消歧可能也会有一些帮助。

  我搜索了一些其他在直接在DOM Tree上利用路径、token进行聚类的论文,接下来几天认
  真调研直接在DOM Tree上做的方法。

\end{document}