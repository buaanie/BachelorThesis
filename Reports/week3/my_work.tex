% Created 2013-03-13 Wed 14:29
\documentclass[a4paper]{article}
\usepackage[utf8]{inputenc}
\usepackage[T1]{fontenc}
\usepackage{fixltx2e}
\usepackage{graphicx}
\usepackage{longtable}
\usepackage{float}
\usepackage{wrapfig}
\usepackage{soul}
\usepackage{textcomp}
\usepackage{marvosym}
\usepackage[nointegrals]{wasysym}
\usepackage{latexsym}
\usepackage{amssymb}
\usepackage{hyperref}
\tolerance=1000
\usepackage{fontspec}
\usepackage{xunicode}
\usepackage{xltxtra}
\usepackage{xeCJK}
\usepackage{listings}
\usepackage{xcolor}
\usepackage{fancyhdr}
\usepackage{fancybox}
\usepackage{comment}
\usepackage{enumerate}
\usepackage{colortbl}
\usepackage{framed}
\usepackage{amsmath}
\usepackage{algorithm}
\usepackage{algorithmic}

\setmainfont{Times New Roman}
\setmonofont{Courier New}
\setCJKmainfont[BoldFont=YouYuan]{SimSun}
\setCJKfamilyfont{song}{SimSun}
\setCJKfamilyfont{msyh}{微软雅黑}
\setCJKfamilyfont{fs}{FangSong}

\lstset{frame=single}

% new and renew command
\newcommand{\reffig}[1]{Figure~\ref{#1}}
\newcommand{\reftbl}[1]{Table~\ref{#1}}
\renewcommand{\contentsname}{目录}
\renewcommand{\baselinestretch}{1.2}

% In case you need to adjust margins:
\topmargin=-0.0in      %
\evensidemargin=0.5in     %
\oddsidemargin=0.5in      %
\textwidth=5.5in        %
\textheight=8.5in       %
\headsep=0.25in         %

\providecommand{\alert}[1]{\textbf{#1}}

\title{毕设工作}
\author{丘骏鹏}
\date{2013-03-13 Wed}
\hypersetup{
  pdfkeywords={},
  pdfsubject={},
  pdfcreator={Emacs Org-mode version 7.8.11}}

\begin{document}

\maketitle



\section{工作综述}
\label{sec-1}

我对主要工作的理解是:从海量的网页数据中,利用大数据的冗余性发现网页模板,抽取结
构化的信息。主要的步骤:
\begin{itemize}
\item 首先需要将目录页和详细页分开,可以根据url特征和简单的文本特征来区别。这部分应
  该要比较简单迅速。
\item 工作的重点在于如何对海量的网页进行聚类和模板抽取。这部分工作应该要充分考虑数据
  的冗余性,利用这些冗余信息来发现不同网页之间的共同点,从而自动生成网页抽取的模
  板。同时考虑到数据量很大,算法不能太复杂。
\end{itemize}
\section{目前的工作进展}
\label{sec-2}

前两周的工作报告放在了report\_{}week1-2.zip中。

这周到目前截止的工作:
\begin{itemize}
\item 阅读了TEXT: Automatic Template Extraction from Heterogeneous Web Pages。这篇论
  文将DOM Tree用路径进行表示,用MDL(Minimun Description Length Principle)作为标
  准进行聚类。由于直接计算MDL复杂度很高,该论文提出一种扩展的MinHash算法,用于近
  似估计MDL,提高算法效率,
\item 阅读了黄老师给的Webpage Understanding: Beyond Page-Level Search。里面主要介绍的
  是如何将webpage understanding拆分成几个子任务。其中提到了一种分割HTML网页的
  VIPS(VIsion-basd Page Segmentation)方法,即将原网页按照视觉区域的分块解析成一
  个vision tree来进行表示。
\item VIPS: a Vision-based Page Segmentation Algorithm(在读)。即上面提到的VIPS算法
  具体实现的论文。我初看的结果是好像算法比较复杂,不确定是否适合应用到大数据上面。
\item \href{http://code.google.com/p/cx-extractor/}{http://code.google.com/p/cx-extractor/} 上的工具和文档。该工具主要就利用了HTML
  文档行块的分布来提取正文,方法和实现都很简单。
\end{itemize}
\section{我对工作的理解}
\label{sec-3}

  工作的重点应该是网页的聚类和模板抽取。

  对于如何进行聚类。我觉得聚类方法可以采用现有的那些聚类算法,关键是如何有效地计
  算两个文档的相似度。HTML可以解析成DOM Tree,可以直接利用这个树表示计算编辑距离
  来衡量相似度,也可以将树通过某种遍历方法转化为其他数据结构再进行计算,或者是用
  路径集合表示,也可以按照VIPS将网页划分成vision-tree后再做计算。我现在还没有想到
  一个很好的计算方法。

  对于模板如何抽取。如果聚类的时候利用了DOM Tree的结构特征,比如标签,路径,模板
  就自动可以表示出来了。如果利用纯内容特征(应该不会这样做),可能会涉及到如何进
  行正则表达式推导问题。

\end{document}