% Created 2013-04-12 Fri 22:10
\documentclass[a4paper]{article}
\usepackage[utf8]{inputenc}
\usepackage[T1]{fontenc}
\usepackage{fixltx2e}
\usepackage{graphicx}
\usepackage{longtable}
\usepackage{float}
\usepackage{wrapfig}
\usepackage{soul}
\usepackage{textcomp}
\usepackage{marvosym}
\usepackage[nointegrals]{wasysym}
\usepackage{latexsym}
\usepackage{amssymb}
\usepackage{hyperref}
\tolerance=1000
\usepackage{fontspec}
\usepackage{xunicode}
\usepackage{xltxtra}
\usepackage{xeCJK}
\usepackage{listings}
\usepackage{xcolor}
\usepackage{fancyhdr}
\usepackage{fancybox}
\usepackage{comment}
\usepackage{enumerate}
\usepackage{colortbl}
\usepackage{framed}
\usepackage{amsmath}
\usepackage{algorithm}
\usepackage{algorithmic}

\setmainfont{Times New Roman}
\setmonofont{Courier New}
\setCJKmainfont[BoldFont=YouYuan]{SimSun}
\setCJKfamilyfont{song}{SimSun}
\setCJKfamilyfont{msyh}{微软雅黑}
\setCJKfamilyfont{fs}{FangSong}

\lstset{frame=single}

% new and renew command
\newcommand{\reffig}[1]{Figure~\ref{#1}}
\newcommand{\reftbl}[1]{Table~\ref{#1}}
\renewcommand{\contentsname}{目录}
\renewcommand{\baselinestretch}{1.2}

% In case you need to adjust margins:
\topmargin=-0.0in      %
\evensidemargin=0.5in     %
\oddsidemargin=0.5in      %
\textwidth=5.5in        %
\textheight=8.5in       %
\headsep=0.25in         %

\providecommand{\alert}[1]{\textbf{#1}}

\title{第七周工作报告}
\author{丘骏鹏}
\date{2013-04-12 Fri}
\hypersetup{
  pdfkeywords={},
  pdfsubject={},
  pdfcreator={Emacs Org-mode version 7.9.4}}

\begin{document}

\maketitle

\setcounter{tocdepth}{3}
\tableofcontents
\vspace*{1cm}


\section{本周工作}
\label{sec-1}

\begin{itemize}
\item 优化原有的算法,做了很多的近似,目前运行时间从2s降到了0.1到0.2s。主要的方法是:
\begin{enumerate}
\item 去掉了结束标签,只保留开始标签,这一步速度提高很多,因为目前LCS算法的复杂度
     是O(mn),所以如果去掉结束标签理论上复杂度变成\(\frac{1}{4}\)。
\item 将字符串做了hash,避免字符串比较,直接比较int数值。这一步速度提升有限。
\item 做了空间上的一些优化,空间复杂度降到O(2*m),这主要是考虑到之后如果载入很多文
     档,内存占用也会很大,做一些这方面的优化也是有必要的。
\end{enumerate}
虽然目前运行速度加快了10倍左右,但是感觉提高的空间不会太大了(至少从算法上来
  说)。如果从语言角度考虑的话,可能的方法是使用C语言实现一个LCS算法,然后用jni来
  调用,可能可以加快一些执行速度(是否有效取决于jni调用多花的时间和C程序执行加速
  的时间之间的大小关系)
\item 学习了Actor模型,基于\href{http://akka.io}{akka} 库实现了一个简单的并行计算的框架。Actor模型中每个
  actor是一个基本的计算单元,互相之间通过传递异步的消息来进行决策,是比线程、锁这
  些并行计算的原语更高层次的一个轻量级的抽象并行计算模型。akka库目前工业上也有较
  广泛的应用,其中包括amazon这些大的公司。我打算采用这个并行框架实现多线程的计算。
\end{itemize}
\section{目前的困难}
\label{sec-2}

\begin{enumerate}
\item 拿最小的博客集合来做实验的话,共60000篇文档。假设目前采用一种简单的聚类算法,
     只需要每两篇文档计算一次距离即可,时间是\(60000^2/2*0.1/3600\)小时,远超可以
     承受的范围,所以必须采用并行的计算方法。如果有20台计算机可以同时计算,每台
     上可以开10个线程,实际需要的时间为大概5000小时。

     和郝老师讨论的方案是先将样本数减少,因为这是和样本个数平方成正比的,所以效
     果会很明显,但是会不会数据量一小就和选题的着重点(大数据量)有些偏差?

     也和房磊学长讨论了一下,他说暂时没有想到很好的解决方案,不过可以放到hadoop
     上做。

     我想大部分的聚类算法可能都需要至少计算一次两两的距离,如果可以显著减少计算相
     似度的次数会提高算法的执行速度,不知道老师能不能提供关于这方面的算法的一些参
     考(或者组里有没有师兄师姐做过类似的工作)?除了减少样本量和计算次数外,只能
     将计算相似度的方法再进行简化。实际在做了多次简化后,目前这个LCS已经算挺粗糙
     的了,我觉得如果采用jaccard或者转化为一个tag vector计算欧几里得距离之类的方
     法就可能太过于简单了(不过时间也可以显著减小)。
\item 聚类的意义。取了些样本做实验,发现实验结果差异较大的是404网页和详细页,意义
     不大。
     
     关于聚类的意义,我的看法是这样的:大数据的优点是可以让我们更好地发现
     “共同点”,但是为了计算速度,我们可能采取较粗糙的算法,这样会导致发现出来的
     “共同点”可能会很稀疏,也就是说模板会很简单。因此如果网页的模板差异不是特别
     大,差异部分可能在这些稀疏的模板体现不明显,也就很难通过聚类区分开来。实际上
     因为我们不知道聚出来的类的个数,我们只有通过控制阈值的方法来控制聚类的过程,
     能分多少类也会取决于我们的阈值是如何取的,如果阈值设置不合适,可能就会有不合
     理的结果。但这个阈值取多少合适本身也很难决定,没有一个好的方法衡量多大的阈值
     是好的。

     另外,进行聚类需要耗费的计算资源这么大,聚类结果实际上可能只会有少量甚至只有
     1类,而且还会因为不同的阈值和聚类的方法存在诸多不确定性,因此这样的做法是否
     真正切合实际的需求?从实用角度考虑,在我们这个系统中这样做是不是有些复杂了?
\end{enumerate}

\end{document}
