
%%% Local Variables:
%%% mode: latex
%%% TeX-master: ../main
%%% End:
% \secretlevel{绝密} \secretyear{2100}

\ctitle{大数据环境下信息抽取模板自动聚类与发现}
% 根据自己的情况选,不用这样复杂
\makeatletter
\ifthu@bachelor\relax\else
  \ifthu@doctor
    \cdegree{工学博士}
  \else
    \ifthu@master
      \cdegree{工学硕士}
    \fi
  \fi
\fi
\makeatother


\cdepartment[计算机]{计算机科学与技术系}
\cmajor{计算机科学与技术}
\cauthor{丘骏鹏} 
\csupervisor{朱小燕}
% 如果没有副指导老师或者联合指导老师,把下面两行相应的删除即可。
\cassosupervisor{郝宇}
% \ccosupervisor{某某某教授}
% 日期自动生成,如果你要自己写就改这个cdate
%\cdate{\CJKdigits{\the\year}年\CJKnumber{\the\month}月}

% 博士后部分
% \cfirstdiscipline{计算机科学与技术}
% \cseconddiscipline{系统结构}
% \postdoctordate{2009年7月——2011年7月}

% \etitle{An Introduction to \LaTeX{} Thesis Template of Tsinghua University} 
% 这块比较复杂,需要分情况讨论:
% 1. 学术型硕士
%    \edegree:必须为Master of Arts或Master of Science(注意大小写)
%              “哲学、文学、历史学、法学、教育学、艺术学门类,公共管理学科
%               填写Master of Arts,其它填写Master of Science”
%    \emajor:“获得一级学科授权的学科填写一级学科名称,其它填写二级学科名称”
% 2. 专业型硕士
%    \edegree:“填写专业学位英文名称全称”
%    \emajor:“工程硕士填写工程领域,其它专业学位不填写此项”
% 3. 学术型博士
%    \edegree:Doctor of Philosophy(注意大小写)
%    \emajor:“获得一级学科授权的学科填写一级学科名称,其它填写二级学科名称”
% 4. 专业型博士
%    \edegree:“填写专业学位英文名称全称”
%    \emajor:不填写此项
% \edegree{Doctor of Engineering} 
% \emajor{Computer Science and Technology} 
% \eauthor{Xue Ruini} 
% \esupervisor{Professor Zheng Weimin} 
% \eassosupervisor{Chen Wenguang} 
% 这个日期也会自动生成,你要改么?
% \edate{December, 2005}

% 定义中英文摘要和关键字
\begin{cabstract}
  随着互联网的发展,研究人员可以获得越来越多的数据,我们已经进入了“大数据”的时
  代。在这样一个背景下,为了能使用计算机更高效地处理这些数据,我们需要从非结构化
  或者半结构化的数据中提取出我们关心的信息,并将其用结构化信息的方式存储下来。目
  前,互联网上的网页大多是通过模板动态生成的,为了从某些类似的网页中提取出结构化
  的信息,我们可以挖掘这类网页的共同点,找出这类网页的模板,然后用模板去抽取网页
  中的信息。

  本文设计并实现了具有以下的功能的系统:对于海量的由各种网页组成的数据,先利用后
  缀树高效地找出每个网页中的重复记录并将其合并,然后通过聚类将不同模板生成的网页
  分开,再从每个类别中利用无监督方法抽取出对应的模板,利用这些模板去抽取我们需要
  的数据。
\end{cabstract}

\ckeywords{大数据, 结构化数据, 后缀树, 聚类, 模板, 无监督学习}

\begin{eabstract} 
  As the growth of the Internet, researchers now could obtain more and more
  data. We are in the era of ``Big Data''. Under such circumstances, we need to
  extract the information we're concerned with from the unstructured or
  semi-structured data and store it in the form of structured data so that
  computers can handle the data more effectively. currently, most of the web
  pages in the Internet are generated by templates. In order to extract
  structured data from these pages, we could dig into the common points of them
  and find out the template of a certain kind of web pages. Then we could make
  use of these templates to extract information from other web pages.

  In this paper, we design and implement a system with following functions: for
  a large set of different web pages, it will first remove all the duplicate
  data records using a data structure called ``suffix tree''. Then it will
  separate web pages which are generated by different templates through
  clustering, and for every cluster, it will generate a corresponding template
  by unsupervised learning. Finally, it could extract the information we need
  from other web pages using the generated templates.
\end{eabstract}

\ekeywords{Big Data, structured data, suffix tree, clustering, template,
  unsupervised learning}
